\documentclass{article}

% \usepackage[english]{babel}
\usepackage[brazilian]{babel}
\usepackage[letterpaper,top=2cm,bottom=2cm,left=3cm,right=3cm,marginparwidth=1.75cm]{geometry}

% Useful packages
\usepackage{amsmath}
\usepackage{graphicx}
% \usepackage[colorlinks=true,linkcolor=blue]{hyperref}
\usepackage{hyperref}
\usepackage{listings}
\usepackage[figure,table,lstlisting]{totalcount}
\usepackage{tcolorbox}
\usepackage{listings}
\usepackage{xcolor}

\hypersetup{
    colorlinks=true,
    linkcolor=black,
    filecolor=magenta,      
    urlcolor=cyan,
    pdftitle={Overleaf Example},
    pdfpagemode=FullScreen,
    }

\definecolor{verde}{rgb}{0.25,0.5,0.35}
\definecolor{jpurple}{rgb}{0.5,0,0.35}

\lstset{
  language=C++,
  basicstyle=\ttfamily\small,
  keywordstyle=\color{jpurple}\bfseries,
  stringstyle=\color{red},
  commentstyle=\color{verde},
  morecomment=[s][\color{blue}]{/**}{*/},
  extendedchars=true,
  showspaces=false,
  showstringspaces=false,
  numbers=left,
  numberstyle=\tiny,
  breaklines=true,
  backgroundcolor=\color{cyan!10},
  breakautoindent=true,
  captionpos=b,
  xleftmargin=0pt,
  tabsize=4
}

\renewcommand{\lstlistingname}{Código}
\renewcommand{\lstlistlistingname}{Lista de Códigos Fonte}

\newcommand\conditionalLoF{%
    \iftotalfigures
        \listoffigures
    \fi}
\newcommand\conditionalLoT{%
    \iftotaltables
        \listoftables
    \fi}
\newcommand\conditionalLoL{%
    \iftotallstlistings
        \lstlistoflistings
    \fi}

\newcommand{\DESCRICAO}[1]{
    % set flexible interword space
    \setlength{\spaceskip}{0.5em plus 1em minus 0.1em}%
    % add some space with not as much flexibility, but only
    % if some space precedes
    \ifdim\lastskip>0pt \hspace{0.5em plus 0.5em minus 0.1em}\fi
    \texttt{\textbf{\color{red}#1}}
}

\newcommand{\CAPA}[6]{
    % 1) Título
    % 2)
    % 3)
    % 4)
    \begin{titlepage}
    	
    	  \vfill
    	
    	  \begin{center}
    	    \begin{large}
    	      Universidade Federal de Ouro Preto - UFOP
    	    \end{large}
    	  \end{center}
    	
    	  \begin{center}
    	    \begin{large}
    	      Instituto de Ciências Exatas e Biológicas - ICEB
    	    \end{large}
    	  \end{center}
    	
    	  \begin{center}
    	    \begin{large}
    	      Departamento de Computação - DECOM
    	    \end{large}
    	  \end{center}
          
          \begin{center}
    	    \begin{large}
              Ciência da Computação
    	    \end{large}
    	  \end{center}
    	  
    	  \vfill
          \vfill
    	
    	  \begin{center}
    	    \begin{Huge}
    	      #1\\
    	    \end{Huge}
    	    \begin{Large}
              #2\\
    	    \end{Large}
    	  \end{center}
    	  
    	  \vfill
    	  
    	  \begin{center}
    	    \begin{large}
              #3
    	    \end{large}
    	  \end{center}

        \begin{center}
    	    \begin{large}
              #4
    	    \end{large}
    	  \end{center}

        \begin{center}
    	    \begin{large}
              #5
    	    \end{large}
    	  \end{center}
    	
    	  \begin{center}
    	    \begin{large}
    	      Professor: #6
    	    \end{large}
    	  \end{center}
    	
    	  \vfill
          \vfill
    	
    	  \begin{center}
    	    \begin{large}
    	      Ouro Preto \\
    	      \today \\
    	    \end{large}
    	  \end{center}    
    
    
    \clearpage
    \newpage

    \tableofcontents
    \conditionalLoF
    \conditionalLoT
    \conditionalLoL
	
    % \thispagestyle{empty}
    % \tableofcontents
    % \newpage
    % \thispagestyle{empty}
    
    % \listoffigures
    
    % \lstlistoflistings
    % \newpage
    % \thispagestyle{empty}
    \end{titlepage}
}

\begin{document}

\CAPA{Trabalho Prático I}{BCC202 - Estruturas de Dados I}{Vitor Oliveira Diniz}{Maria Luiza Aragão}{Jessica Machado}{Pedro Silva}



\section{Introdução}

Neste trabalho foi necessário entregar o código em C e um relatório referente ao que foi desenvolvido. O algoritmo a ser desenvolvido é um modelo de autômato celular, mais especificamente o jogo da vida (the game of life).

A codificação foi feita em C, usando somente a biblioteca padrão da GNU, sem o uso de bibliotecas adicionais.


\subsection{Especificações do problema}

Autômatos celulares são simples problemas de auto reprodução celular em que o sistema baseia-se em estados anteriores.  Um autômato é definido por seu espaço celular e sua regra de transição. Nosso espaço celular é composto por um retículo de N células idênticas, que se encontram mortas ou vivas em um arranjo bi-dimensional. Assim, devemos implementar um programa em linguagem C que nos permita ler um reticulado do jogo da vida e retornar a malha da próxima geração, assim seguindo as regras definidas pelo jogo. Essas que consistem em:

Assim, devemos implementar um programa em linguagem C que nos permita ler um reticulado do jogo da vida e retornar a malha da próxima geração, assim seguindo as regras definidas pelo jogo. Essas que consistem em:

\begin{itemize}
    \item Manter uma célula viva caso duas ou três de suas células vizinhas também estiverem vivas;
    \item Células vivas morrem caso haja superpopulação de três células vizinhas também vivas;
    \item Células vivas morrem caso haja menos de duas células vivas vizinhas;
    \item Células mortas tornam-se vivas caso haja três células vizinhas vivas.
\end{itemize}

\subsection{Considerações Iniciais}
Algumas ferramentas foram utilizadas durante a criação deste projeto:

\begin{itemize}
  \item Ambiente de desenvolvimento do código fonte: Visual Studio Code.
  \item Linguagem utilizada: C.
  \item Ambiente de desenvolvimento da documentação: Visual Studio Code \LaTeX Workshop.
\end{itemize}

\subsection{Ferramentas utilizadas}
Algumas ferramentas foram utilizadas para testar a implementação, como:

\begin{itemize}
    \item[-] \textit{CLANG}: ferramentas de análise estática do código.
    \item[-] \textit{Valgrind}: ferramentas de análise dinâmica do código.
\end{itemize}

\subsection{Especificações da máquina}
A máquina onde o desenvolvimento e os testes foram realizados possui a seguinte configuração:

\begin{itemize}
    \item[-] Processador: Ryzen 7-5800H.
    \item[-] Memória RAM: 16 Gb.
    \item[-] Sistema Operacional: Arch Linux x86\_64.
\end{itemize}


\subsection{Instruções de compilação e execução}

Para a compilação do projeto, basta digitar:

\begin{tcolorbox}[title=Compilando o projeto,width=\linewidth]
    gcc -c automato.c -Wall \newline
    gcc -c tp.c -Wall \newline
    gcc  -g automato.o tp.o -o exe
\end{tcolorbox}

Usou-se para a compilação as seguintes opções:

\begin{itemize}
    \item [-] \emph{-g}: para compilar com informação de depuração e ser usado pelo Valgrind.
    \item [-] \emph{-Wall}: para mostrar todos os possível \emph{warnings} do código.
    \item [-] \emph{-c}: Para compilar o arquivo sem linkar os arquivos para obtermos um arquivo do tipo objeto.
\end{itemize}

Para a execução do programa basta digitar:
\begin{tcolorbox}[title=,width=\linewidth]
    ./exe $<$ caminho\_até\_o\_arquivo\_de\_entrada
\end{tcolorbox}

Onde “caminho-até-o-arquivo-de-entrada” pode ser: “1.in” para realizar o primeiro caso de teste e “2.in” para realizar o segundo.

\clearpage



\section{Desenvolvimento}

Seguindo as boas práticas de programação, implementamos um tipo abstrato de dados ( TAD ) para a representação do nosso problema. De acordo com o pedido, e para uma melhor organização, o nosso código foi modularizado em três arquivos, tp.c automato.h e automato.c em que o arquivo tp.c deve apenas invocar e tratar as respostas das funções e procedimentos definidos no arquivo automato.h.

\subsection{TAD}
Para começar a resolução do problema proposto, decidimos criar uma struct, que representaria o nosso autômato celular e seria nosso TAD, que no caso possui dois estados, vivo ou morto. Esses dois estados foram representados por um enum, ou enumeração, que é um tipo de dado definido pelo programador que irá assumir apenas um dos valores definidos.Seguindo essa lógica, determinamos o valor 0 para MORTO e 1 para VIVO.

\begin{lstlisting}[caption={TAD representando um automato celular},label={lst:cod1},language=C]

    typedef enum
{
    MORTO = 0,
    VIVO = 1
} VIDA;

struct automatoCelular
{
    VIDA vida;
};

\end{lstlisting}

\subsection{Funções}

A seguir entraremos em detalhe sobre as principais funções utilizadas no programa.

\subsubsection{lerDimensao}

Criamos essa função para armazenar a dimensão da tabela em um ponteiro, passado através de um parâmetro.

\begin{lstlisting}[caption={Função lerDimensao},label={lst:cod2},language=C]

void lerDimensao(int *dimensao)
{
    scanf("%d", dimensao);
}

\end{lstlisting}

\subsubsection{alocarReticulado}

Posteriormente criamos esta função para alocar dinamicamente a tabela com as dimensões inseridas, primeiramente alocamos um ponteiro de tamanho n do nosso autômato, que será correspondente a cada linha. E usamos uma repetição para alocar n elementos em cada linha. Foi utilizada a função malloc da biblioteca padrão que realiza essa alocação dinamicamente.

\begin{lstlisting}[caption={Função alocarReticulado},label={lst:cod3},language=C]

AutomatoCelular **alocarReticulado(int dimensao)
{
    AutomatoCelular **automatoCelular = malloc(dimensao * sizeof(AutomatoCelular *));
    
    for (int i = 0; i < dimensao; i++)
    {
        automatoCelular[i] = malloc(dimensao * sizeof(AutomatoCelular));
    }
    
    return automatoCelular;
    
    }
    
    \end{lstlisting}

\subsubsection{desalocarReticulado}
Esta função foi implementada com o objetivo de liberar o espaço de memória que alocamos para a tabela. Como alocamos uma matriz, foi necessário o uso de uma estrutura de repetição.

\begin{lstlisting}[caption={Função desalocarReticulado},label={lst:cod4},language=C]
void desalocarReticulado(AutomatoCelular ***automatoCelular, int dimensao)
{
    for (int i = 0; i < dimensao; i++)
    {
        free((*automatoCelular)[i]);
    }
    free(*automatoCelular);
}

\end{lstlisting}


\subsubsection{leituraReticulado}

Aqui usamos como parâmetro a nossa matriz e sua dimensão, para assim poder utilizá-las nas repetições que possibilitariam o preenchimento do usuário em relação aos valores da tabela. Criamos uma variável do tipo inteiro para receber esses valores e na repetição atribuímos ele a matriz na posição atual, avançando para a próxima posição a cada vez que o “for” se repetir.

\begin{lstlisting}[caption={Função leituraReticulado},label={lst:cod5},language=C]
void leituraReticulado(AutomatoCelular **automatoCelular, int dimensao)
{
    int vidaReticulado;

    for (int i = 0; i < dimensao; i++)
    {
        for (int j = 0; j < dimensao; j++)
        {
            scanf("%d", &vidaReticulado);
            automatoCelular[i][j].vida = vidaReticulado;
        }
    }
}

\end{lstlisting}

\subsubsection{copiaAutomato}

Esta função de cópia foi implementada com o intuito de ser uma cópia da matriz original já que, no nosso código, contamos as células ao redor e atualizamos sua situação (viva ou morta) imediatamente, o que poderia afetar no resultado final.

\begin{lstlisting}[caption={Função copiaAutomato},label={lst:cod6},language=C]

AutomatoCelular **copiaAutomato(AutomatoCelular **automatoCelular, int dimensao){


    AutomatoCelular **copia;
    copia = alocarReticulado(dimensao);

    for(int i = 0; i < dimensao; i++){
        for(int j = 0; j < dimensao; j++){
            copia[i][j].vida = automatoCelular[i][j].vida; 
        }
    }

    return copia;

}

\end{lstlisting}


\subsubsection{evoluirReticulado}

Para a função de evoluir o reticulado, iremos dividir a abordagem em 3 partes.

Primeiramente, devemos percorrer cada célula do reticulado	para assim aplicarmos as regras definidas pelo jogo da vida.
Atenção para a cópia da nossa matriz de autómatos que foi feita.

\begin{lstlisting}[caption={Função evoluirReticulado parte 1},label={lst:cod7},language=C]
AutomatoCelular **evoluirReticulado(AutomatoCelular **automatoCelular, int dimensao)
{
    
    AutomatoCelular **CopiaAutomatoCelular;
    CopiaAutomatoCelular = copiaAutomato(automatoCelular, dimensao);
    
    for (int i = 0; i < dimensao; i++)
    {
        for (int j = 0; j < dimensao; j++)
        {
    
\end{lstlisting}

Em todas as regras devemos percorrer as células adjacentes a que está sendo analisada, assim devemos definir o valor inicial e final do for, caso algum elemento esteja na borda, não queremos acessar uma posição não alocada da matriz, então faremos um simples tratamento de exceção em que a posição inicial e final sempre estarão alocadas.

\begin{lstlisting}[caption={Função evoluirReticulado parte 2},label={lst:cod8},language=C]

    int contadorCelular = 0;
    int x0 = i - 1;
    int y0 = j - 1;
    int xf = i + 1;
    int yf = j + 1;

    
    if (x0 < 0) 
        x0 = 0;

    if (y0 < 0) 
        y0 = 0;

    if(xf >= dimensao)
        xf = dimensao - 1;

    if(yf >= dimensao)
        yf = dimensao - 1;


\end{lstlisting}

Agora contaremos todas as células vivas adjacentes à célula analisada para enfim decidirmos qual regra será aplicada, já que todas as regras se baseiam na quantidade de células vivas adjacentes.

\begin{lstlisting}[caption={Função evoluirReticulado parte 3},label={lst:cod9},language=C]

    for (int k = x0; k <= xf; k++)
    {
        for (int l = y0; l <= yf; l++)
        {
            if (!(k == i && l == j))
            {
                // printf("checando a celula[%d][%d] = %d\n",k, j, automatoCelular[k][l].vida);
                if (CopiaAutomatoCelular[k][l].vida == VIVO)
                {
                    contadorCelular++;
                }
            }
        }
    }

\end{lstlisting}

em seguida iremos aplicar as regras do jogo da vida através de simples condições que dependem do estado da célula e do numero de células vivas adjacentes

\begin{lstlisting}[caption={Função evoluirReticulado parte 4},label={lst:cod10},language=C]
    
    //celula renasce
    if (automatoCelular[i][j].vida == MORTO && contadorCelular == 3)
    {
        automatoCelular[i][j].vida = VIVO;
    } // solidao
    else if (automatoCelular[i][j].vida == VIVO && contadorCelular < 2)
    {
        automatoCelular[i][j].vida = MORTO;
    } // continua viva
    else if (automatoCelular[i][j].vida == VIVO && (contadorCelular == 2  || contadorCelular == 3))
    {
        automatoCelular[i][j].vida = VIVO;
    } //sufocamento
    else if (automatoCelular[i][j].vida == VIVO && contadorCelular > 3)
    {
        automatoCelular[i][j].vida = MORTO;
    }

\end{lstlisting}


Ao final da função, iremos desalocar nossa matriz de cópia e retornar a nova geração do automato.	

\begin{lstlisting}[caption={Função evoluirReticulado parte 5},label={lst:cod11},language=C]
    
    desalocarReticulado(&CopiaAutomatoCelular, dimensao);
    return automatoCelular;
}

\end{lstlisting}

\clearpage
\subsection{main}

Na função main invocamos as funções necessárias para a realização dos procedimentos, sendo eles: a leitura dos dados da matriz, a sua alocação, seu tratamento, sua impressão e por último, desalocação.

\begin{lstlisting}[caption={função main},label={lst:cod11},language=C]

#include <stdio.h>
#include <stdlib.h>
#include "automato.h"

int main(){
    int dimensao;
    AutomatoCelular **automatoCelular;

    lerDimensao(&dimensao);
    automatoCelular = alocarReticulado(dimensao);
    leituraReticulado(automatoCelular, dimensao);
    automatoCelular = evoluirReticulado(automatoCelular, dimensao);
    imprimeReticulado(automatoCelular, dimensao);
    desalocarReticulado(&automatoCelular, dimensao);

    return 0;
}

\end{lstlisting}

\clearpage


\section{Impressões Gerais}

Com as funções já pré definidas foi muito mais fácil construir a lógica para o desenvolvimento modular do código. O nosso grupo então se reuniu e pensou coletivamente sobre a ordem de execução das funções e suas utilidades. O uso do Latex para a documentação foi um ponto positivo para o trabalho, com um membro do grupo já tendo certos conhecimentos e o restante aprendendo com o decorrer do uso.
Outro conhecimento adquirido e posto em prática foi o uso do TAD, tipo abstrato de dados, para a solução do nosso problema. Houve também o desenvolvimento de um código bem modularizado, com uma excelente ajuda das instruções contidas no documento que nos foi disponibilizado como exemplo.

\section{Análise}

Após o desenvolvimento do programa, a primeira análise feita foi atráves dos casos de teste disponibilizados na página do trabalho prático do run.codes, com simples exemplos de entrada e saída, executamos o programa com um dos exemplos de entrada e assim, foi possível fazer uma simples análise se o programa se comportava corretamente. Primeiramente obtivemos algumas dificuldades que serão abordadas na conclusão. Após a correção do erro encontrado, as próximas realizações de testes apresentaram resultados iguais ao exemplo de saída disponibilizado.
Depois dos testes iniciais para verificar um funcionamento inicial do programa, utilizamos o valgrind, uma ferramenta de análise dinâmica de código para conferir se há algum memory leak ou warning referente a manipulação de memória.

Após esses dois testes, fizemos o envio para o run codes que apresenta uma bateria de teste ainda mais extensa e obtivemos o resultado esperado.

\section{conclusão}

Com este trabalho ampliamos os nossos conhecimentos referente a criação de um TAD ( tipo abstrato de dados ) e como aplicá-lo para encontrar a solução de um problema. Aprendemos como utilizar e fazer um documento em LaTeX, que será extremamente útil e essencial ao longo do curso, tanto para apresentação de trabalhos acadêmicos e para o desenvolvimento de artigos científicos.
Como uma dificuldade inicial, tivemos o erro de interpretação em relação aos dados da matriz e suas modificações.

Estávamos realizando a leitura das células adjacentes e aplicando as regras de transição imediatamente, considerando os novos dados. Após um tempo pensando sobre o que poderia estar afetando os nossos resultados, chegamos à conclusão de que os dados que deveriam ser levados em consideração seriam os da matriz original e não da matriz modificada gradativamente ao decorrer do programa.

\end{document}