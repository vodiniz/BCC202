\documentclass{article}

\input{setup}

\begin{document}

\CAPA{Trabalho Prático II}{BCC202 - Estruturas de Dados I}{Vitor Oliveira Diniz}{Maria Luiza Aragão}{Jéssica Machado}{Pedro Silva}



\section{Introdução}

Neste trabalho foi necessário entregar o código em C e um relatório referente ao que foi desenvolvido. O algoritmo a ser desenvolvido é uma ordenação de objetos móveis com base em sua trajetória.

A codificação foi feita em C, usando somente a biblioteca padrão da GNU, sem o uso de bibliotecas adicionais.


\subsection{Especificações do problema}

No contexto de objetos móveis, as trajetórias são um conjunto de pontos das posições ocupadas pelo objeto durante seu movimento. 
Uma trajetória pode ser representada em função de uma sequência de dois ou mais pontos com coordenadas (x,y) distintas. Assim, devemos, com base em uma lista de objetos móveis e o total de pontos de cada trajetória,  ambos quantativamente desconhecidos, 
ordenar a lista de saída nos baseando nas seguintes regras:

\begin{itemize}
    \item decrescentemente, com base na distância percorrida;
    \item crescentemente, com base no o deslocamento e nome, respectivamente, caso haja empate no item anterior.
\end{itemize}

\subsection{Considerações Iniciais}
Algumas ferramentas foram utilizadas durante a criação deste projeto:

\begin{itemize}
  \item Ambiente de desenvolvimento do código fonte: Visual Studio Code.
  \item Linguagem utilizada: C.
  \item Ambiente de desenvolvimento da documentação: Visual Studio Code \LaTeX Workshop.
\end{itemize}

\subsection{Ferramentas utilizadas}
Algumas ferramentas foram utilizadas para testar a implementação, como:

\begin{itemize}
    \item[-] \textit{CLANG}: ferramentas de análise estática do código.
    \item[-] \textit{Valgrind}: ferramentas de análise dinâmica do código.
\end{itemize}

\subsection{Especificações da máquina}
A máquina onde o desenvolvimento e os testes foram realizados possui a seguinte configuração:

\begin{itemize}
    \item[-] Processador: Ryzen 7-5800H.
    \item[-] Memória RAM: 16 Gb.
    \item[-] Sistema Operacional: Arch Linux x86\_64.
\end{itemize}


\subsection{Instruções de compilação e execução}

Para a compilação do projeto, basta digitar:

\begin{tcolorbox}[title=Compilando o projeto,width=\linewidth]
    gcc tp.o ordenacao.o -o exe -g -Wall -lm\newline
    gcc tp.c -c -g -lm\newline
    gcc ordenacao.c -c -g -lm -Wall\newline
\end{tcolorbox}

Usou-se para a compilação as seguintes opções:

\begin{itemize}
    \item [-] \emph{-g}: para compilar com informação de depuração e ser usado pelo Valgrind.
    \item [-] \emph{-Wall}: para mostrar todos os possível \emph{warnings} do código.
    \item [-] \emph{-c}: Para compilar o arquivo sem linkar os arquivos para obtermos um arquivo do tipo objeto.
    \item [-] \emph{-lm}: Para fazer o link da biblioteca math.h
    \item [-] \emph{-o}: Para indicar o nome do arquivo de saida.

\end{itemize}

Para a execução do programa basta digitar:
\begin{tcolorbox}[title=,width=\linewidth]
    ./exe $<$ caminho\_até\_o\_arquivo\_de\_entrada
\end{tcolorbox}

Onde “caminho-até-o-arquivo-de-entrada” pode ser: “1.in” para realizar o primeiro caso de teste e “2.in” para realizar o segundo.

\clearpage



\section{Desenvolvimento}

Seguindo as boas práticas de programação, implementamos um tipo abstrato de dados 
( TAD ) para a representação do nosso problema. De acordo com o pedido, e para uma 
melhor organização, o nosso código foi modularizado em três arquivos, tp.c ordenacao.h e
ordenacao.c em que o arquivo tp.c deve apenas invocar e tratar as respostas das funções e 
procedimentos definidos no arquivo ordenacao.h.

\subsection{TAD}
Para começar a resolução do problema proposto, decidimos criar duas structs, 
que representariam o nosso objeto e seriam nosso TAD. O primeiro, struct Ponto, 
que nos informaria, com base no plano cartesiano, as coordenadas do Objeto,
nossa segunda struct, que contém informações necessárias para distinguir um 
objeto do outro, seja sua trajetória, distância e ID.

\begin{lstlisting}[caption={TAD's representando Ponto e Objeto, respectivamente},label={lst:cod1},language=C]

typedef struct{
    int x;
    int y;
}Ponto;

typedef struct{
    char ID[5];
    Ponto *pontos;
    int npontos;
    double distancia;
    double deslocamento;
}Objeto;

\end{lstlisting}

\subsection{Funções}

A seguir entraremos em detalhe sobre as principais funções utilizadas no programa.


\subsubsection{validaEntrada}

Criamos essa função para alocar dinâmicamente, vários pontos, baseado em um valor digitado pelo usuário.

\begin{lstlisting}[caption={Função validaEntrada},label={lst:cod2},language=C]

    void validaEntrada(int *numero_objetos, int *numero_pontos){
    
    //Recebe a quantidade de objetos e de pontos que eles possuem
    scanf("%d %d", numero_objetos, numero_pontos); //Recebe a quantidade de objetos e de pontos que eles possuem


    while(*numero_objetos <= 0 || *numero_pontos <= 0 ){
        printf("O numero de entrada para pontos e objetos e invalido. Por favor digite novamente.\n");
        scanf("%d %d", numero_objetos, numero_pontos); //Recebe a quantidade de objetos e de pontos que eles possuem
    }

}
\end{lstlisting}



\subsubsection{alocaPontos}

Criamos essa função para alocar dinâmicamente, vários pontos, baseado em um valor digitado pelo usuário.

\begin{lstlisting}[caption={Função alocaPontos},label={lst:cod2},language=C]

Ponto* alocaPontos (int npontos){

    Ponto *pontos = (Ponto*)malloc(npontos * sizeof(Ponto));
    return pontos;
}

\end{lstlisting}

\subsubsection{alocaObjetos}

Essa função aloca dinâmicamente um número X de objetos. Em seguida, com a ajuda de
uma estrutura de repetição, adicionamos em cada um dos objetos o número de pontos,
alocamos os pontos e definimos o valor inicial do deslocamento e da distância.

\begin{lstlisting}[caption={Função alocaObjetos},label={lst:cod3},language=C]

Objeto* alocaObjetos (int npontos, int nobj){

    Objeto* objetos = (Objeto*)malloc(nobj * sizeof(Objeto));

    for (int i = 0; i < nobj; i++){

        objetos[i].npontos = npontos;
        objetos[i].pontos = alocaPontos(npontos);

        objetos[i].deslocamento = 0;
        objetos[i].distancia = 0;

    }
    
    return objetos;
}
    
    \end{lstlisting}

\subsubsection{desalocaPontos e desalocaObjetos}
Esta função foi implementada com o objetivo de liberar o espaço de memória que alocamos para o ponto e o objeto.

\begin{lstlisting}[caption={Funções desalocaPontos e desalocaObjetos},label={lst:cod4},language=C]
void desalocaPontos (Ponto *pontos){
    free(pontos);
}
    
void desalocaObjetos(Objeto **lista, int nobj){
    
    for(int i = 0; i < nobj; i++){
        desalocaPontos((*lista)[i].pontos);
        printf("I: %d\n", i);
    }
    free(*lista);
}

\end{lstlisting}


\subsubsection{lerPontos e lerObjetos}

Na função lerPontos, utilizamos uma estrutura de repetição que se repete até o número de pontos (npontos), váriavel passada por parâmetro, e, em cada um dos pontos, pegamos o valor de X e Y.
\\Já na função lerObjetos, a extrutura de repetição aconteceria até o número de objetos (nobj), também passada por parâmetro, e com o scanf pegariamos o ID do objeto e o valor de seus pontos, utilizando a função lerPontos

\begin{lstlisting}[caption={Funções lerPontos e lerObjetos},label={lst:cod5},language=C]
void lerPontos(Ponto* pontos, int npontos){
    for (int i = 0; i < npontos; i++){
        scanf("%d", &pontos[i].x);
        scanf("%d", &pontos[i].y);
    }
}
    
    
void lerObjetos(Objeto *lista, int nobj){
    for(int i = 0; i < nobj; i++){
        scanf("%s", lista[i].ID);
        lerPontos(lista[i].pontos, lista->npontos);
    }
        
}

\end{lstlisting}

\subsubsection{calcularDistancia e calcularDeslocamento}

Para o cálculo dessas funções, utilizamos as fórmulas, já conhecidas, da distância e do deslocamento, e as funções da biblioteca math.h, que facilitou muito nossas contas.

\begin{lstlisting}[caption={Funções calcularDistancia e calcularDeslocamento},label={lst:cod6},language=C]

double calcularDistancia (Objeto *objeto){
    double distancia = 0;
    for(int i = 0; i < objeto->npontos - 1; i++){
        distancia += sqrt(pow((objeto->pontos[i+1].x - objeto->pontos[i].x), 2) + pow((objeto->pontos[i+1].y - objeto->pontos[i].y), 2));
    }
    return distancia ;
}
    
double calcularDeslocamento (Objeto *objeto){
    return sqrt(pow((objeto->pontos[objeto->npontos-1].x - objeto->pontos[0].x), 2) 
+ pow((objeto->pontos[objeto->npontos-1].y - objeto->pontos[0].y), 2));
}

\end{lstlisting}


\subsubsection{realizaCalculos}

Utilizando uma estrutura de repetição, mudamos os valores da distância e do deslocamento de cada um dos objetos utilizando as funções calcularDistancia e calcularDeslocamento, anteriormente explicadas.

\begin{lstlisting}[caption={Função realizaCalculos},label={lst:cod7},language=C]
void realizaCalculos(Objeto *objetos, int nobj){
    for (int i = 0; i < nobj; i++)
    {
        objetos[i].distancia = calcularDistancia(&objetos[i]);
        objetos[i].deslocamento = calcularDeslocamento(&objetos[i]);
    }
   
}
    
\end{lstlisting}

\subsubsection{comparaObjeto}

Função necessária para fazer a comparação necessária de nossos TADs, seguindo as regras
definidas no trabalho. Optamos por fazer uma função separada 
para que o código ficasse mais legível e modularizado. 


\begin{lstlisting}[caption={Função comparaObjetos},label={lst:cod7},language=C]

int comparaObjeto(Objeto *objeto1, Objeto *objeto2){

    if (definitelyGreaterThan(objeto1->distancia, objeto2->distancia)){
        return 0;
        
    } else if (definitelyLessThan(objeto1->distancia, objeto2->distancia)){
        return 1;
    } else {

        if (definitelyGreaterThan(objeto1->deslocamento, objeto2->deslocamento)) {
            return 1;

        } else if(definitelyLessThan(objeto1->deslocamento, objeto2->deslocamento)){
            return 0;
        } else{
            if(strcmp(objeto1->ID, objeto2->ID) > 0){
                return 1;
            } else {
                return 0;
            }
        }
        
    }
}
    
\end{lstlisting}

\subsubsection{shellSort}

Para ordenação, optamos por utilizar o método shellSort.

\begin{lstlisting}[caption={Função shellSort},label={lst:cod7},language=C]

void shellSort(Objeto *objetos, int n) {
    int h = 1;
    Objeto aux;
    while( h < n)
        h = 3 * h + 1;

    do{

        h = (h - 1)/3;

        for ( int i = h; i < n;i++){
            aux = objetos[i]; 
            int j = i;
            while(comparaObjeto(&objetos[j - h], &aux)){
                objetos[j] = objetos[j-h];
                j = j - h;
                if ( j < h){
                    break;
                }
            }
            objetos[j] = aux;
            
        }
    } while ( h != 1); 
}
    
\end{lstlisting}


\subsubsection{mergeSort}

Implementamos também a função mergeSorte, que faz uma chamada recursiva dela mesma
no passo da divisão, para depois chamarmos a merge no passo da conquista e fazermos a ordenação.

\begin{lstlisting}[caption={Função mergeSort},label={lst:cod7},language=C]

void mergesort(Objeto *v, int l, int r, int npontos){
    if (l < r){
        int m = (l + r)/2;
        mergesort(v, l, m, npontos);
        mergesort(v, m + 1, r, npontos);
        merge(v, l, m, r, npontos);

    }
}

\end{lstlisting}



\subsubsection{merge}

Implementamos a função merge, que será responsável pela parte da conquista, em que após a divisão
irá ordenar os subvetores em um maior, até que nosso vetor esteja completamente ordenado.

\begin{lstlisting}[caption={Função merge},label={lst:cod7},language=C]

    void merge(Objeto *v, int l, int m, int r, int npontos){

    int size_l = (m - l + 1);
    int size_r = (r - m);

    Objeto *vet_l = malloc( size_l * sizeof(Objeto));
    Objeto *vet_r = malloc( size_r * sizeof(Objeto));


    for (int i = 0; i < size_l; i++)
        vet_l[i] = v[i + l];


    for (int j = 0; j < size_r; j++)
        vet_r[j] = v[m + j + 1];



    int i = 0; 
    int j = 0;

    for (int k = l; k <= r; k++){
         
        if (i == size_l)
            v[k] = vet_r[j++];
        
        else if (j == size_r)
            v[k] = vet_l[i++];

        else if (comparaObjeto(&vet_l[i], &vet_r[j]))
            v[k] = vet_r[j++];

        else
            v[k] = vet_l[i++];
    }

    free(vet_l);
    free(vet_r);

}
    
\end{lstlisting}
    

\clearpage
\subsection{main}

Na função main, além de utilizarmos o scanf para obter o nobj e npontos, invocamos as funções necessárias para a realização dos procedimentos, sendo eles: a alocação dos objetos, a leitura deles, os cálculos necessários, a ordenação da lista, sua impressão e por último, desalocação.

\begin{lstlisting}[caption={função main},label={lst:cod11},language=C]

#include <stdio.h>
#include "ordenacao.h"

int main(void){

    Objeto *lista; //Cria a lista do tipo Objeto
    int npontos, nobj;

    validaEntrada(&nobj, &npontos);
    lista = alocaObjetos(nobj, npontos); //Lista recebe sua alocacao



    if (lista ==NULL){
        printf("Falha na alocacao do vetor. Saindo do programa");
        return 0;
    }
    lerObjetos(lista, nobj); //Chama a funcao de leitura das informacoes de cada objeto


    realizaCalculos(lista, nobj); //Realiza todos os calculos necessarios
    mergesort(lista, 0, nobj - 1, npontos); //Chama o metodo de ordenacao
    // shellSort(lista, nobj);
    imprime(lista, nobj); //Imprime as informacoes
    desalocaObjetos(&lista, nobj); //Desaloca toda memoria reservada

    return 0;
    }
\end{lstlisting}

\clearpage


\section{Impressões Gerais}

Primeiramente além das funções necessárias passadas no PDF, pensamos em quais funções a mais
deveriamos implementar. Com as funções já pré definidas foi muito mais fácil construir 
a lógica para o desenvolvimento modular do código. O nosso grupo então se
reuniu e pensou coletivamente sobre a ordem de execução das funções e suas 
utilidades.Outro conhecimento adquirido e posto em prática foi o uso de 
diferentes métodos de ordenação, para a solução do nosso problema.
Inicialmente, obtivemos um pouco de dificuldade na implementação do MergeSort e acabamos optando
pela implementação do ShellSort, porém depois com um pouco de dificuldade conseguimos resolver o problema
e optamos por deixar os dois métodos no trabalho.
Tambêm tivemos um leve problema com a comparação de números de pontos flututantes, em que foi necessário uma pequena
pausa no trabalho para que pudessemos pensar em uma solução. Foi ai que surgiu nossas funções de comparação para doubles.
Assim houve também o desenvolvimento de um código bem modularizado, 
com uma excelente ajuda das instruções contidas no documento que nos foi disponibilizado como exemplo.

\section{Análise}

Após o desenvolvimento do programa, a primeira análise feita 
foi atráves dos casos de teste disponibilizados na página do 
trabalho prático do run.codes, com simples exemplos de entrada
e saída, executamos o programa com um dos exemplos de entrada
e assim, foi possível fazer uma simples análise se o programa
se comportava corretamente. Obtivemos algumas dificuldades
que serão abordadas na conclusão. Após a implementação
de outro método de ordenação, visto que o primeiro não
foi eficiente, as próximas realizações de testes apresentaram 
resultados iguais ao exemplo de saída disponibilizado.
Depois dos testes iniciais para verificar um funcionamento 
inicial do programa, utilizamos o valgrind, uma ferramenta 
de análise dinâmica de código para conferir se há algum memory 
leak ou warning referente a manipulação de memória.


\section{Conclusão}

Com este trabalho ampliamos os nossos conhecimentos referente 
aos métodos de ordenação, em especial o shellSort e mergeSort, e como aplicá-lo
para encontrar a solução de um problema. 
Como uma dificuldade inicial, tivemos a implementação do mergeSort 
que não gerou os resultados esperados, além de vários erros quando 
rodamos o valgrind. Esse erros eram devidos a liberação errônea do vetor auxiliar
de objetos utilizados na função merge. Pois como faziamos uma atribuição de ponteiros
o conteúdo dele não era copiado, apenas o ponteiro e ao realizar a liberação dele
no final da função, era liberado o conteudo dos pontos causando um erro de double free e leitura inválida.
Após um certo tempo, desistimos e utilizamos o shellSort. Porém após algum
tempo pensando no problema, descobrimos nosso erro e decidimos reimplementar o mergeSort.

\end{document}